\documentclass[ignorenonframetext,]{beamer}
\usetheme{Madrid}
\usecolortheme{dolphin}
\usefonttheme{serif}
\usepackage{amssymb,amsmath}
\usepackage{ifxetex,ifluatex}
\usepackage{fixltx2e} % provides \textsubscript
\usepackage{lmodern}
\ifxetex
  \usepackage{fontspec,xltxtra,xunicode}
  \defaultfontfeatures{Mapping=tex-text,Scale=MatchLowercase}
  \newcommand{\euro}{€}
\else
  \ifluatex
    \usepackage{fontspec}
    \defaultfontfeatures{Mapping=tex-text,Scale=MatchLowercase}
    \newcommand{\euro}{€}
  \else
    \usepackage[T1]{fontenc}
    \usepackage[utf8]{inputenc}
      \fi
\fi
\IfFileExists{upquote.sty}{\usepackage{upquote}}{}
% use microtype if available
\IfFileExists{microtype.sty}{\usepackage{microtype}}{}

% Comment these out if you don't want a slide with just the
% part/section/subsection/subsubsection title:
\AtBeginPart{
  \let\insertpartnumber\relax
  \let\partname\relax
  \frame{\partpage}
}
\AtBeginSection{
  \let\insertsectionnumber\relax
  \let\sectionname\relax
  \frame{\sectionpage}
}
\AtBeginSubsection{
  \let\insertsubsectionnumber\relax
  \let\subsectionname\relax
  \frame{\subsectionpage}
}

\setlength{\parindent}{0pt}
\setlength{\parskip}{6pt plus 2pt minus 1pt}
\setlength{\emergencystretch}{3em}  % prevent overfull lines
\setcounter{secnumdepth}{0}
\usepackage{amsmath, commath, mathtools}
\usepackage{physics}
\graphicspath{img}

\title{Uncertainty Principle}
\author{Daniel Wysocki and Kenny Roffo}
\date{March 3, 2015}

\begin{document}
\frame{\titlepage}

\begin{frame}{Generalized Uncertainty Principle}

\begin{itemize}
\item
  for any observable \(A\), the variance can be expressed by

  \begin{displaymath}
  \sigma_A^2=\braket{(\hat{A}-\expval{A})\Psi}{(\hat{A}-\expval{A})\Psi}
  \end{displaymath}
\item
  we define \(f:=(\hat{A}-\expval{A})\Psi\), and
  \(g:=(\hat{B}-\expval{B})\Psi\) for any other observable \(B\)
\item
  multiplying the variance of two observables we have

  \begin{displaymath}
  \sigma_A^2\sigma_B^2=\braket{f}\braket{g}\ge \abs{\braket{f}{g}}^2
  \end{displaymath}
\item
  for any complex number \(z\), we have

  \begin{displaymath}
  \abs{z}^2=[\Re(z)]^2+[\Im(z)]^2\ge[\Im(z)]^2=\left[\frac{1}{2i}(z-z^*)\right]^2
  \end{displaymath}
\end{itemize}

\end{frame}

\begin{frame}{Generalized Uncertainty Principle}

\begin{itemize}
\itemsep1pt\parskip0pt\parsep0pt
\item
  now if we let \(z=\braket{f}{g}\) we have

  \begin{displaymath}
  \sigma_A^2\sigma_B^2\ge\left(\frac{1}{2i}[\braket{f}{g}-\braket{g}{f}]\right)^2
  \end{displaymath}
\end{itemize}

\begin{align*}
\braket{f}{g} &= \braket{(\hat{A}-\expval{A})\Psi}{(\hat{B}-\expval{B})\Psi}
\\
&=\braket{\Psi}{(\hat{A}-\expval{A})(\hat{B}-\expval{B})\Psi}
\\
&=\braket{\Psi}{(\hat{A}\hat{B}-\hat{A}\expval{B}-\hat{B}\expval{A}+\expval{A}\expval{B})\Psi}
\\
&=\braket{\Psi}{\hat{A}\hat{B}\Psi}-\expval{B}\braket{\Psi}{\hat{A}\Psi}-\expval{A}\braket{\Psi}{\hat{B}\Psi}+\expval{A}\expval{B}\braket{\Psi}{\Psi}
\\
&=\expval{\hat{A}\hat{B}}-\expval{B}\expval{A}-\expval{A}\expval{B}+\expval{A}\expval{B}
\\
&=\expval{\hat{A}\hat{B}}-\expval{A}\expval{B}
\end{align*}

\end{frame}

\begin{frame}{Generalized Uncertainty Principle}

\begin{itemize}
\item
  thus \(\braket{f}{g}=\expval*{\hat{A}\hat{B}}-\expval{A}\expval{B}\)
  and by the same process
  \(\braket{g}{f}=\expval*{\hat{B}\hat{A}}-\expval{B}\expval{A}\)
\item
  so

  \begin{displaymath}
  \braket{f}{g}-\braket{g}{f}=\expval*{\hat{A}\hat{B}}-\expval*{\hat{B}\hat{A}}=\expval*{[\hat{A},\hat{B}]}
  \end{displaymath}
\item
  therefore we have the general uncertainty principle:

  \begin{displaymath}
  \sigma_A^2\sigma_B^2\ge\left(\frac{1}{2i}\expval*{[\hat{A},\hat{B}]}\right)^2
  \end{displaymath}
\end{itemize}

\end{frame}

\begin{frame}{Uncertainty Principle of Momentum and Position}

\begin{itemize}
\itemsep1pt\parskip0pt\parsep0pt
\item
  consider the Hermitian operators \(\hat{x}\) and \(\hat{p}\) whose
  conjugate is \([\hat{x},\hat{p}]=i\hbar\)
\end{itemize}

\begin{align*}
\sigma_x^2\sigma_p^2 &\ge \left(\frac{1}{2i}\expval{[\hat{x},\hat{p}]}\right)^2
\\
\sigma_x^2\sigma_p^2 &\ge \left(\frac{1}{2i}i\hbar\right)^2
\\
\sigma_x^2\sigma_p^2 &\ge \left(\frac{\hbar}{2}\right)^2
\\
\sigma_x\sigma_p &\ge \frac{\hbar}{2}
\end{align*}

\end{frame}

\begin{frame}{Incompatible Observables}

\begin{itemize}
\itemsep1pt\parskip0pt\parsep0pt
\item
  there is an ``uncertainty principle'' for every pair of observables
  \(A\) and \(B\) such that \(\comm*{A}{B} \neq 0\)

  \begin{itemize}
  \itemsep1pt\parskip0pt\parsep0pt
  \item
    incompatible observables
  \end{itemize}
\item
  they cannot have a complete set of common eigenfunctions
\item
  in contrast, \emph{compatible} observables do have complete sets of
  simultaneous eigenfunctions, and therefore have uncertainty principles
  which take the form \[\sigma_A^2 \sigma_B^2 \ge 0\]
\end{itemize}

\end{frame}

\begin{frame}{Compatible Observables Example}

\begin{itemize}
\itemsep1pt\parskip0pt\parsep0pt
\item
  for example, we will find \(\sigma_T \sigma_p\), where \(T\) and \(p\)
  are the kinetic energy and momentum, respectively
\item
  \(\hat{T} = \hat{p}^2 / 2m\)

  \begin{align*}
    \comm*{\hat{T}}{\hat{p}} &=
    \hat{T}\hat{p} - \hat{p}\hat{T}
    \\ &=
    \frac{\hat{p}^2}{2 m} \hat{p} - \hat{p} \frac{\hat{p}^2}{2 m}
    \\ &=
    \frac{\hat{p}^3}{2 m} - \frac{\hat{p}^3}{2 m} = 0
    \\ \implies
    \sigma_T \sigma_p \geq 0
  \end{align*}
\end{itemize}

\end{frame}

\begin{frame}{The Minimum-Uncertainty Wave Packet}

\ldots{}

\end{frame}

\begin{frame}{The Energy--Time Uncertainty Principle}

\begin{itemize}
\itemsep1pt\parskip0pt\parsep0pt
\item
  the position--momentum uncertainty principle is often written as
  \[\Delta x \Delta p \geq \frac{\hbar}{2}\]
\item
  it is often accompanied by the energy--time uncertainty principle
  \[\Delta t \Delta E \geq \frac{\hbar}{2}\]
\item
  don't be fooled, they may look similar but are entirely different
\item
  position, momentum, and energy are all dynamical variables, while time
  is an independent variable
\item
  now we will work towards deriving it
\end{itemize}

\end{frame}

\begin{frame}{The Energy--Time Uncertainty Principle}

\begin{itemize}
\itemsep1pt\parskip0pt\parsep0pt
\item
  we begin by computing the time derivative of the expectation value of
  an observable, \(Q(x, p, t)\)

  \begin{align*}
  \dod{}{t} \expval{Q} &=
  \dod{}{t} \braket*{\Psi}{\hat{Q} \Psi}
  \\ &=
  \braket*{\dpd{\Psi}{t}}{\hat{Q} \Psi} +
  \braket*{\Psi}{\dpd{\hat{Q}}{t} \Psi} +
  \braket*{\Psi}{\hat{Q} \dpd{\Psi}{t}}
  \end{align*}
\item
  use the Schrödinger equation to substitute for \(\pd{\Psi}{t}\)
  \[\imath\hbar \dpd{\Psi}{t} = \hat{H} \Psi\]
\end{itemize}

\end{frame}

\begin{frame}{The Energy--Time Uncertainty Principle}

\[\dod{}{t} \expval{Q} =
  -\frac{1}{\imath\hbar} \braket*{\hat{H} \Psi}{\hat{Q} \Psi} =
  +\frac{1}{\imath\hbar} \braket{\Psi}{\hat{Q} \hat{H} \Psi}
  +\expval{\dpd{\hat{Q}}{t}}\]

\begin{itemize}
\itemsep1pt\parskip0pt\parsep0pt
\item
  \(\hat{H}\) is hermitian, so
  \(\braket*{\hat{H} \Psi}{\hat{Q} \Psi} = \braket*{\Psi}{\hat{H}\Hat{Q}\Psi}\)
\end{itemize}

\[\dod{}{t} \expval{Q} =
  \frac{\imath}{\hbar} \expval{\comm*{\hat{H}}{\hat{Q}}}
+ \expval{\dpd{\hat{Q}}{t}}\]

\begin{itemize}
\itemsep1pt\parskip0pt\parsep0pt
\item
  if \(\hat{Q}\) is time-independent, the rate of change of the
  expectation value is determined by the commutator of \(\hat{Q}\) with
  \(\hat{H}\)
\item
  \(\comm*{\hat{H}}{\hat{Q}} = 0 \implies \expval{Q}\) is constant
\end{itemize}

\end{frame}

\begin{frame}{The Energy--Time Uncertainty Principle}

\begin{itemize}
\itemsep1pt\parskip0pt\parsep0pt
\item
  using \(A = H\) and \(B = Q\) in the generalized uncertainty
  principle, and assuming \(Q\) is time-independent, we see that
  \[\sigma_H^2 \sigma_Q^2 \geq
    \qty( \frac{1}{2 \imath} \expval*{\comm*{\hat{H}}{\hat{Q}}} )^2 =
    \qty( \frac{1}{2 \imath} \frac{\hbar}{\imath} \dod{\expval*{Q}}{t} )^2 =
    \qty( \frac{\hbar}{2} )^2 \qty( \dod{\expval*{Q}}{t} )^2\]
  \[\sigma_H \sigma_Q \geq \frac{\hbar}{2} \abs{\dod{\expval*{Q}}{t}}\]
  \[\Delta E := \sigma_H,
    \text{ and }
    \Delta t := \frac{\sigma_Q}{\abs{\dif\expval{Q}/\dif t}}\]
  \[\Delta E \Delta t \geq \frac{\hbar}{2}\]
\end{itemize}

\end{frame}

\begin{frame}{The Energy--Time Uncertainty Principle}

\begin{itemize}
\itemsep1pt\parskip0pt\parsep0pt
\item
  \(\Delta t\) represents the amount of time it takes the expectation
  value of \(Q\) to change by one standard deviation
  \[\sigma_Q = \abs{\dod{\expval*{Q}}{t}} \Delta t\]
\item
  \(\Delta t\) depends on the observable, \(Q\), being observed
\item
  if \(\Delta E\) is small, the rate of change for \emph{all}
  observables must be small
\item
  if \emph{any} observable changes rapidly, the ``uncertainty'' in the
  energy must be large
\end{itemize}

\end{frame}

\end{document}
